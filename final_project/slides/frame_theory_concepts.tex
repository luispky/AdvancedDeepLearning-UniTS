\documentclass[12pt]{article}
\usepackage{amsmath, amssymb, amsthm}
\usepackage{fullpage}
\usepackage{mathtools}
\usepackage{physics}
\usepackage{bm}

\title{Formal Proofs and Structure of Frame Operators, Tight Frames, and Equiangular Frames}
\author{}
\date{}

\newtheorem{definition}{Definition}
\newtheorem{theorem}{Theorem}
\newtheorem{lemma}{Lemma}
\newtheorem{proposition}{Proposition}
\newtheorem{remark}{Remark}
\newtheorem{corollary}{Corollary}

\begin{document}

\maketitle

\section{Introduction}

In this document, we rigorously develop the structure of frame operators in finite-dimensional Hilbert spaces, prove the tight frame condition via matrix identities, and define equiangular frames and their relation to the Gram matrix. Throughout, we consider $\mathcal{H} = \mathbb{R}^m$ for simplicity.

\section{Frames in Hilbert Spaces}

\begin{definition}[Frame]
A collection of vectors $\{f_i\}_{i=1}^n \subset \mathcal{H}$ is called a \textbf{frame} for $\mathcal{H}$ if there exist constants $A, B > 0$ such that for all $x \in \mathcal{H}$,
\[
A \|x\|^2 \leq \sum_{i=1}^n |\langle x, f_i \rangle|^2 \leq B \|x\|^2.
\]
The constants $A$ and $B$ are called the \emph{frame bounds}.

Let $\mathcal{H} = \mathbb{R}^m$ and $X \in \mathbb{R}^{m \times n}$ be the matrix whose $i$-th column is $f_i$, i.e., $X = [f_1 \; f_2 \; \dots \; f_n]$. Then the frame condition can be written as:
\[
A \|x\|^2 \leq x^T X X^T x \leq B \|x\|^2.
\]
This follows because $\sum_{i=1}^n |\langle x, f_i \rangle|^2 = \sum_{i=1}^n (x^T f_i)^2 = x^T \left(\sum_{i=1}^n f_i f_i^T\right) x = x^T X X^T x$.

\end{definition}

\begin{definition}[Frame Operator]
Given a frame $\{f_i\}_{i=1}^n$, the \textbf{frame operator} $S: \mathcal{H} \to \mathcal{H}$ is defined by
\[
Sx = \sum_{i=1}^n \langle x, f_i \rangle f_i.
\]
\end{definition}

\section{Outer Product Decomposition of the Frame Operator}

Let us represent each $f_i \in \mathbb{R}^m$ and define the matrix $X \in \mathbb{R}^{m \times n}$ whose $i$-th column is $f_i$, i.e.,
\[
X = [f_1 \; f_2 \; \dots \; f_n].
\]

We now prove that the frame operator can be written as a sum of outer products.

\begin{proposition}[Frame Operator as Outer Product Sum]
\[
S = XX^T = \sum_{i=1}^n f_i f_i^T.
\]
\end{proposition}

\begin{proof}
First, recall that the standard basis vector $e_i \in \mathbb{R}^n$ satisfies $(e_i)_j = \delta_{ij}$. The matrix $X$ can be written as:
\[
X = \sum_{i=1}^n f_i e_i^T.
\]
This is because $f_i e_i^T$ produces a matrix with $f_i$ in the $i$-th column and zeros elsewhere.

Now compute the frame operator:
\begin{align*}
XX^T &= \left( \sum_{i=1}^n f_i e_i^T \right) \left( \sum_{j=1}^n f_j e_j^T \right)^T \\
&= \left( \sum_{i=1}^n f_i e_i^T \right) \left( \sum_{j=1}^n e_j f_j^T \right) \\
&= \sum_{i=1}^n \sum_{j=1}^n f_i (e_i^T e_j) f_j^T \\
&= \sum_{i=1}^n f_i f_i^T.
\end{align*}
\end{proof}

\section{Explicit Matrix Representations}

\subsection{Outer Product Matrix \( f_i f_i^T \)}

Let \( f_i \in \mathbb{R}^m \) be:
\[
f_i = \begin{bmatrix} f_{i1} \\ f_{i2} \\ \vdots \\ f_{im} \end{bmatrix},
\quad
f_i^T = \begin{bmatrix} f_{i1} & f_{i2} & \cdots & f_{im} \end{bmatrix}.
\]

Then the outer product \( f_i f_i^T \in \mathbb{R}^{m \times m} \) is:
\[
f_i f_i^T =
\begin{bmatrix}
f_{i1}^2 & f_{i1}f_{i2} & \cdots & f_{i1}f_{im} \\
f_{i2}f_{i1} & f_{i2}^2 & \cdots & f_{i2}f_{im} \\
\vdots & \vdots & \ddots & \vdots \\
f_{im}f_{i1} & f_{im}f_{i2} & \cdots & f_{im}^2
\end{bmatrix}.
\]

\subsection{Frame Operator \( XX^T \)}

Let \( X = [f_1 \; f_2 \; \dots \; f_n] \in \mathbb{R}^{m \times n} \). Then:
\[
XX^T = \sum_{i=1}^n f_i f_i^T.
\]

Each entry of \( XX^T \) can be written in coordinates as:
\[
(XX^T)_{jk} = \sum_{i=1}^n f_{ij} f_{ik}, \quad 1 \leq j,k \leq m.
\]

- Diagonal entries: \( (XX^T)_{jj} = \sum_{i=1}^n f_{ij}^2 \)
- Off-diagonal entries: \( (XX^T)_{jk} = \sum_{i=1}^n f_{ij} f_{ik} \)

\subsection{Example: \( m = 3, n = 2 \)}

Let:
\[
f_1 = \begin{bmatrix} a_1 \\ b_1 \\ c_1 \end{bmatrix},
\quad
f_2 = \begin{bmatrix} a_2 \\ b_2 \\ c_2 \end{bmatrix}.
\]

Then:
\[
XX^T = f_1 f_1^T + f_2 f_2^T =
\begin{bmatrix}
a_1^2 + a_2^2 & a_1 b_1 + a_2 b_2 & a_1 c_1 + a_2 c_2 \\
a_1 b_1 + a_2 b_2 & b_1^2 + b_2^2 & b_1 c_1 + b_2 c_2 \\
a_1 c_1 + a_2 c_2 & b_1 c_1 + b_2 c_2 & c_1^2 + c_2^2
\end{bmatrix}.
\]

This matrix is symmetric and encodes the total energy and pairwise correlations of the components of the frame vectors.

\subsection{Geometric Interpretation: Energy and Correlations}

The structure of the frame operator $XX^T$ provides crucial geometric insight into what it means for a frame to be tight or equiangular:

\begin{itemize}
\item \textbf{Energy Distribution (Diagonal Entries):} The diagonal entries $(XX^T)_{jj} = \sum_{i=1}^n f_{ij}^2$ represent the total energy contributed by all frame vectors to the $j$-th coordinate direction. In a tight frame, this energy is uniformly distributed across all coordinate directions.

\item \textbf{Cross-Correlations (Off-Diagonal Entries):} The off-diagonal entries $(XX^T)_{jk} = \sum_{i=1}^n f_{ij} f_{ik}$ capture the correlation between different coordinate directions induced by the frame vectors. These correlations reveal how the frame vectors couple different dimensions of the space.

\item \textbf{Tight Frame Property:} When $XX^T = AI_m$, the frame achieves perfect energy balance: each coordinate direction receives exactly the same total energy $A$, and all cross-correlations vanish. This means the frame vectors collectively treat all coordinate directions symmetrically, providing optimal stability for reconstruction.

\item \textbf{Equiangular Property:} In an equiangular frame, the pairwise inner products $\langle f_i, f_j \rangle$ for $i \neq j$ all have the same absolute value. This constraint on the Gram matrix $X^TX$ propagates to the frame operator $XX^T$, creating a specific pattern of correlations that balances the geometric arrangement of the frame vectors.
\end{itemize}

The interplay between these properties becomes particularly elegant when a frame is both tight and equiangular: the frame operator achieves perfect energy balance while maintaining optimal angular separation between frame elements.

\section{Tight Frames}

\begin{definition}[Tight Frame]
A frame $\{f_i\}_{i=1}^n$ is called a \textbf{tight frame} if the frame bounds satisfy $A = B$. In this case, the frame inequality becomes:
\[
\sum_{i=1}^n |\langle x, f_i \rangle|^2 = A \|x\|^2 \quad \forall x \in \mathcal{H}.
\]
\end{definition}

\begin{proposition}
Let $\{f_i\}_{i=1}^n$ be a tight frame in $\mathbb{R}^m$, and let $X \in \mathbb{R}^{m \times n}$ be the matrix of frame vectors. Then:
\[
XX^T = A I_m.
\]
\end{proposition}

\begin{proof}
Let $S = XX^T$ be the frame operator. Then by definition,
\[
Sx = \sum_{i=1}^n \langle x, f_i \rangle f_i.
\]
Since the frame is tight with bound $A$, we know:
\[
\langle Sx, x \rangle = \sum_{i=1}^n |\langle x, f_i \rangle|^2 = A \|x\|^2 = \langle A x, x \rangle.
\]
Thus, for all $x \in \mathbb{R}^m$, we have:
\[
\langle (S - A I)x, x \rangle = 0.
\]
This implies $S = A I_m$ by the polarization identity and positive-definiteness of $S$.
\end{proof}

\begin{corollary}
If the frame vectors are unit norm and the frame is tight, then
\[
A = \frac{n}{m}.
\]
\end{corollary}

\begin{proof}
Taking the trace of both sides:
\[
\text{Tr}(XX^T) = \sum_{i=1}^n \text{Tr}(f_i f_i^T) = \sum_{i=1}^n \|f_i\|^2 = n,
\]
while
\[
\text{Tr}(XX^T) = \text{Tr}(A I_m) = A \cdot m.
\]
Equating: \( A m = n \Rightarrow A = \frac{n}{m} \).
\end{proof}

\section{Equiangular Frames}

\begin{definition}[Equiangular Frame]
A frame $\{f_i\}_{i=1}^n$ of unit-norm vectors is called \textbf{equiangular} if there exists a constant $\alpha \in [0,1]$ such that for all $i \ne j$,
\[
|\langle f_i, f_j \rangle| = \alpha.
\]
\end{definition}

\begin{remark}
The equiangular property implies that the off-diagonal entries of the Gram matrix $G = X^T X$ all have the same absolute value.
\end{remark}

\begin{proposition}
If a unit-norm frame is equiangular and tight, then its Gram matrix $G = X^T X \in \mathbb{R}^{n \times n}$ satisfies:
\begin{itemize}
    \item $G_{ii} = 1$ for all $i$ (unit norm)
    \item $|G_{ij}| = \alpha$ for all $i \ne j$
    \item $\text{rank}(G) = m$
\end{itemize}
\end{proposition}

\begin{proof}
- Each diagonal entry is $G_{ii} = \langle f_i, f_i \rangle = \|f_i\|^2 = 1$.
- For $i \ne j$, by the equiangular condition, $|\langle f_i, f_j \rangle| = \alpha$.
- Since $G = X^T X$, its rank is equal to the rank of $X$, which is $m$ (assuming the frame spans $\mathbb{R}^m$).
\end{proof}

\end{document}
